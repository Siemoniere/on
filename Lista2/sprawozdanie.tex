\documentclass{article}
\usepackage[utf8]{inputenc}
\usepackage[T1]{fontenc}
\usepackage{polski}
\usepackage{amsmath}
\usepackage{geometry}
\usepackage{float}
\usepackage{graphicx}

\geometry{a4paper, margin=2.5cm}

\title{Obliczenia naukowe: Sprawozdanie 1}
\author{Szymon Hładyszewski}
\date{\today}

\begin{document}

\maketitle

\section{Zadanie 1}
\subsection{Opis problemu}
Zadanie polegało na powtórzeniu obliczeń z zadania 5 z listy 1, ale bez ostatnich cyfr w $x_4$ i $x_5$. Poniżej znadują się dwa oryginalne wektory:
\begin{itemize}
    \item $ x = [2.718281828, -3.141592654, 1.414213562, 0.5772156649, 0.3010299957] $
    \item $ y = [1486.2497, 878366.9879, -22.37492, 4773714.647, 0.000185049] $
\end{itemize}
Dane zaburzonego wektora x:
\begin{itemize}
    \item $ x_{noised} = [2.718281828, -3.141592654, 1.414213562, 0.577215664, 0.301029995] $
\end{itemize}
Obliczenia polegały na wyznaczeniu iloczynu skalarnego tych wektorów czterema sposobami:
\begin{enumerate}
    \item w przód
    \item w tył
    \item od największego do najmniejszego
    \item od najmniejszego do największego
\end{enumerate}
\subsection{Rozwiązanie}
Kod rozwiązania zadania znajduje się w pliku \texttt{zadanie1.jl}.
\subsection{Wyniki i interpretacja}
Poniższa tabela ilustruje wyniki obliczeń iloczynu skalarnego dla wektorów z pełną precyzją oraz z obciętymi ostatnimi cyframi. Widzimy różnice w wynikach dla Float64 po obcięciu ostatnich cyfr, co wskazuje na wpływ precyzji obliczeń.

\begin{table}[h!]
\centering
\begin{tabular}{lcc}
\hline
\textbf{Przypadek} & \textbf{Pełna wartość} & \textbf{Ucięta wartość} \\
\hline
\multicolumn{3}{c}{\textbf{Float32}} \\
\hline
w przód & -0.4999443 & -0.4999443 \\
w tył & -0.4543457 & -0.4543457 \\
od największego do najmniejszego & -0.5 & -0.5 \\
od najmniejszego do największego & -0.5 & -0.5 \\
\hline
\multicolumn{3}{c}{\textbf{Float64}} \\
\hline
w przód & $1.0251881368\times 10^{-10}$ & -0.00429634274 \\
w tył & $-1.5643308870\times 10^{-10}$ & -0.00429634299 \\
od największego do najmniejszego & 0.0 & -0.00429634284 \\
od najmniejszego do największego & 0.0 & -0.00429634284 \\
\hline
\end{tabular}
\caption{Porównanie wyników iloczynów skalarnych (Float32 i Float64) przed i po zmianie}
\end{table}
\subsection{Wnioski}
Dla Float32 wyniki się nie rożnią, co wynika ze zbyt małej precyzji tej arytmetyki dla tak mało znaczących zmian. Jednakże dla Float64 mimo tych z pozoru nic nieznaczących detali, widzimy niemałe różnice w wynikach. Widać zatem jak bardzo wrażliwe są te algorytmy na nawet najmniejsze zmiany danych.
\section{Zadanie 2}
\subsection{Opis problemu}
Zadanie polegało na znalezieniu granicy funkcji  $f(x) = e^{x} \times ln(1 + e^{-x})$ w nieskończoności oraz zwizualizowaniu wykresu w wybranych dwóch programach do wizualizacji.
\subsection{Rozwiązanie}
Kod rozwiązania zadania znajduje się w pliku \texttt{zadanie2.jl}.
\subsection{Wyniki i interpretacja}
Według programu Julia, granica funkcji $f(x)$ w nieskończoności wynosi $1$. Wykresy funkcji wygenerowane w programach Desmos oraz Wolfram Alpha przedstawiono na rysunkach poniżej. Jak widzimy na poniższych wykresach, znacznie odbiegają one od uzyskanego wyniku.

\begin{figure}[h!]
    \centering
    \includegraphics[width=0.7\textwidth]{wykres1.png}
    \caption{Wykres funkcji $f(x) =e^{x} \times ln(1 + e^{-x})$ w programie Desmos}
    \label{fig:wykres}
\end{figure}
\begin{figure}[h!]
    \centering
    \includegraphics[width=0.7\textwidth]{wykres2.png}
    \caption{Wykres funkcji $f(x) =e^{x} \times ln(1 + e^{-x})$ w programie Wolfram Alpha}
    \label{fig:wykres}
\end{figure}
\subsection{Wnioski}
Jak widać na wykresach, obie wizualizacje zaczynaja dziwnie oscylować w przedziale od 30 do 40, mimo że granica wynosi 1. Wynika to z faktu, że dla dużego x mnożymy bardzo duże $e^{x}$ oraz bardzo małe $ln(1 + e^{-x})$, co prowadzi do dużego znaczenia błędu wynikającego z arytmetyki w stosunku do rzeczywistej wartości. Dla odpowiednio dużego x, $ln(1 + e^{-x})$ zostaje zaokrąglane do 0, przez co wartość funkcji od pewnego momentu również spada do zera.
\section{Zadanie 3}
\subsection{Opis problemu}
Zadanie polegało na wygenerowaniu macierzy A oraz rozwiązaniu układu równań liniowych postaci $Ax = b$ za pomocą dwóch metod:
\begin{itemize}
    \item eliminacja Gaussa 
    \item obliczenie odwrotności macierzy A i skorzystanie ze wzoru $x = A^{-1}b$
\end{itemize}
Macierz A miała być wygenerowana na dwa sposoby:
\begin{itemize}
    \item macierz Hilberta
    \item macierz losowa o zadanym wskaźniku uwarunkowania
\end{itemize}
\subsection{Rozwiązanie}
Kod rozwiązania zadania znajduje się w pliku \texttt{zadanie3.jl}. Funkcje generujące macierz A zostały wzięte ze strony wykładowcy na jego prośbę i zastosowane w tymże programie.
\subsection{Wyniki i interpretacja}
\begin{table}[H]
\centering
\begin{tabular}{cccccc}
\hline
$n$ & errGauss & errInv & $\mathrm{cond}(A)$ & $\mathrm{rank}(A)$ \\
\hline
1 & 0.0 & 0.0 & 1.0 & 1 \\
2 & $5.66\times10^{-16}$ & $1.40\times10^{-15}$ & 19.281 & 2 \\
3 & $8.02\times10^{-15}$ & 0.0 & 524.06 & 3 \\
4 & $4.14\times10^{-14}$ & 0.0 & 1.55$\times10^{4}$ & 4 \\
5 & $1.68\times10^{-12}$ & $3.35\times10^{-12}$ & 4.77$\times10^{5}$ & 5 \\
6 & $2.62\times10^{-10}$ & $2.02\times10^{-10}$ & 1.50$\times10^{7}$ & 6 \\
7 & $1.26\times10^{-8}$ & $4.71\times10^{-9}$ & 4.75$\times10^{8}$ & 7 \\
8 & $6.12\times10^{-8}$ & $3.08\times10^{-7}$ & 1.53$\times10^{10}$ & 8 \\
9 & $3.88\times10^{-6}$ & $4.54\times10^{-6}$ & 4.93$\times10^{11}$ & 9 \\
10 & $8.67\times10^{-5}$ & $2.50\times10^{-4}$ & 1.60$\times10^{13}$ & 10 \\
11 & $1.58\times10^{-4}$ & $7.62\times10^{-3}$ & 5.22$\times10^{14}$ & 10 \\
12 & 0.13396 & 0.25899 & 1.75$\times10^{16}$ & 11 \\
13 & 0.11040 & 5.33128 & 3.19$\times10^{18}$ & 11 \\
14 & 1.45541 & 8.71499 & 6.20$\times10^{17}$ & 11 \\
15 & 4.69667 & 7.34464 & 3.68$\times10^{17}$ & 12 \\
16 & 54.1552 & 29.8488 & 7.05$\times10^{17}$ & 12 \\
17 & 13.7072 & 10.5169 & 1.25$\times10^{18}$ & 12 \\
18 & 10.2576 & 24.7621 & 2.25$\times10^{18}$ & 12 \\
19 & 102.160 & 109.946 & 6.47$\times10^{18}$ & 13 \\
20 & 108.318 & 114.344 & 1.15$\times10^{18}$ & 13 \\
\hline
\end{tabular}
\caption{Wyniki dla macierzy Hilberta}

\end{table}
\begin{table}[H]
\centering
\begin{tabular}{ccccc}
\hline
$\mathrm{cond}(A)$ & $n$ & errGauss & errInv & $\mathrm{rank}(A)$ \\
\hline
1        & 5  & $1.79\times10^{-16}$ & $9.93\times10^{-17}$ & 5 \\
1        & 10 & $3.49\times10^{-16}$ & $2.19\times10^{-16}$ & 10 \\
1        & 20 & $5.13\times10^{-16}$ & $4.29\times10^{-16}$ & 20 \\
\hline
10       & 5  & $3.18\times10^{-16}$ & $2.05\times10^{-16}$ & 5 \\
10       & 10 & $6.27\times10^{-16}$ & $4.85\times10^{-16}$ & 10 \\
10       & 20 & $4.74\times10^{-16}$ & $5.61\times10^{-16}$ & 20 \\
\hline
$10^{3}$ & 5  & $9.01\times10^{-16}$ & $1.18\times10^{-14}$ & 5 \\
$10^{3}$ & 10 & $3.09\times10^{-14}$ & $3.55\times10^{-14}$ & 10 \\
$10^{3}$ & 20 & $5.38\times10^{-14}$ & $5.67\times10^{-14}$ & 20 \\
\hline
$10^{7}$ & 5  & $5.22\times10^{-10}$ & $5.17\times10^{-10}$ & 5 \\
$10^{7}$ & 10 & $2.80\times10^{-10}$ & $3.44\times10^{-10}$ & 10 \\
$10^{7}$ & 20 & $3.96\times10^{-10}$ & $3.51\times10^{-10}$ & 20 \\
\hline
$10^{12}$ & 5  & $8.16\times10^{-6}$  & $8.20\times10^{-6}$  & 5 \\
$10^{12}$ & 10 & $7.89\times10^{-6}$  & $1.14\times10^{-5}$  & 10 \\
$10^{12}$ & 20 & $4.36\times10^{-5}$  & $4.29\times10^{-5}$  & 20 \\
\hline
$10^{16}$ & 5  & 0.18657 & 0.15068 & 4 \\
$10^{16}$ & 10 & $3.72\times10^{-16}$ & 0.05559 & 9 \\
$10^{16}$ & 20 & 0.48812 & 0.48500 & 19 \\
\hline
\end{tabular}
\caption{Błędy względne rozwiązań układów dla macierzy losowych przy rosnącym $\mathrm{cond}(A)$}

\end{table}
Analizując powyższe tabele, widzimy, że wraz ze wzrostem rozmiaru macierzy Hilberta, błąd względny zarówno dla metody eliminacji Gaussa, jak i metody wykorzystującej odwrotność macierzy rośnie znacząco.
\subsection{Wnioski}
W macierzy Hilberta zarówno wskaźnik uwarunkowania jak i błędy względne dla poszczególnych metod rosną bardzo szybko wraz ze wzrostem macierzy. Na podstawie tych obserwacji można stwierdzić, że macierz Hilberta jest źle uwarunkowana. Jeśli chodzi o macierze losowe, widać, że ich wielkość błędu względnego mrtodami Gaussa i inwersji zależy przede wszytskim od wskaźnika uwarunkowania, a nie od rozmiaru.
\section{Zadanie 4}
\subsection{Opis problemu}
Zadanie polegało na znalezieniu 20 miejsc zerowych wielomianu Wilkinsona stosując metodę roots() dla jego postaci naturalnej i sprawdzeniu ich dla tejże postaci oraz iloczynowej i wyjaśnienie rozbieżności. Następnie należało zmienić współczynnik przy $x^{19}$ z $-210$ na $-210 - 2^{-23}$ i wyjaśnić zjawisko.


Oto wielomian Wilkinsona w postaci iloczynowej:
\[P(x) = (x-1)(x-2)(x-3)...(x-20)\]
oraz w postaci naturalnej:
\begin{multline}
p(x) = x^{20} - 210x^{19} + 20615x^{18} - 1256850x^{17} + 53327946x^{16} - 1672280820x^{15} + 40171771630x^{14} \\
{} - 756111184500x^{13} + 11310276995381x^{12} - 135585182899530x^{11} + 1307535010540395x^{10} \\
{} - 10142299865511450x^{9} + 63030812099294896x^{8} - 311333643161390640x^{7} + 1206647803780373360x^{6} \\
- 3599979517947607200x^{5} + 8037811822645051776x^{4} - 12870931245150988800x^{3} + 13803759753640704000x^{2} \\
- 8752948036761600000x + 2432902008176640000
\end{multline}
\subsection{Rozwiązanie}
Kod rozwiązania zadania znajduje się w pliku \texttt{zadanie4.jl}.
\subsection{Wyniki i interpretacja}
Poniższa tabela przedstawia znalezione miejsca zerowe i ich sprawdzenia:
\begin{table}[H]
\centering
\begin{tabular}{cccccc}
\hline
$k$ & $r_k$ & $|P(r_k)|$ & $|p(r_k)|$ & $|r_k - k|$ \\
\hline
1  & 0.9999999999996989  & $3.57\times10^{4}$  & $5.52\times10^{6}$  & $3.01\times10^{-13}$ \\
2  & 2.0000000000283182  & $1.76\times10^{5}$  & $7.38\times10^{19}$ & $2.83\times10^{-11}$ \\
3  & 2.9999999995920965  & $2.79\times10^{5}$  & $3.32\times10^{20}$ & $4.08\times10^{-10}$ \\
4  & 3.9999999837375317  & $3.03\times10^{6}$  & $8.85\times10^{20}$ & $1.63\times10^{-8}$ \\
5  & 5.000000665769791   & $2.29\times10^{7}$  & $1.84\times10^{21}$ & $6.66\times10^{-7}$ \\
6  & 5.999989245824773   & $1.29\times10^{8}$  & $3.32\times10^{21}$ & $1.08\times10^{-5}$ \\
7  & 7.000102002793008   & $4.81\times10^{8}$  & $5.42\times10^{21}$ & $1.02\times10^{-4}$ \\
8  & 7.999355829607762   & $1.64\times10^{9}$  & $8.26\times10^{21}$ & $6.44\times10^{-4}$ \\
9  & 9.002915294362053   & $4.88\times10^{9}$  & $1.20\times10^{22}$ & $2.92\times10^{-3}$ \\
10 & 9.990413042481725   & $1.36\times10^{10}$ & $1.66\times10^{22}$ & $9.59\times10^{-3}$ \\
11 & 11.025022932909318  & $3.59\times10^{10}$ & $2.25\times10^{22}$ & $2.50\times10^{-2}$ \\
12 & 11.953283253846857  & $7.53\times10^{10}$ & $2.89\times10^{22}$ & $4.67\times10^{-2}$ \\
13 & 13.07431403244734   & $1.96\times10^{11}$ & $3.81\times10^{22}$ & $7.43\times10^{-2}$ \\
14 & 13.914755591802127  & $3.58\times10^{11}$ & $4.61\times10^{22}$ & $8.52\times10^{-2}$ \\
15 & 15.075493799699476  & $8.22\times10^{11}$ & $5.90\times10^{22}$ & $7.55\times10^{-2}$ \\
16 & 15.946286716607972  & $1.55\times10^{12}$ & $7.01\times10^{22}$ & $5.37\times10^{-2}$ \\
17 & 17.025427146237412  & $3.69\times10^{12}$ & $8.57\times10^{22}$ & $2.54\times10^{-2}$ \\
18 & 17.99092135271648   & $7.65\times10^{12}$ & $1.01\times10^{23}$ & $9.08\times10^{-3}$ \\
19 & 19.00190981829944   & $1.14\times10^{13}$ & $1.20\times10^{23}$ & $1.91\times10^{-3}$ \\
20 & 19.999809291236637  & $2.79\times10^{13}$ & $1.40\times10^{23}$ & $1.91\times10^{-4}$ \\
\hline
\end{tabular}
\caption{Analiza pierwiastków wielomianu Wilkinsona}

\end{table}
Powyższa tabela pokazuje, że wyliczone pierwiastki wielomianu Wilkinsona różnią się od rzeczywistych wartości, a błąd rośnie wraz z każdym kolejnym pierwiastkiem.
\begin{table}[H]
\centering
\begin{tabular}{cccccc}
\hline
$k$ & $r_k$ & $|P(r_k)|$ & $|p(r_k)|$ & $|r_k - k|$ \\
\hline
1  & 0.9999999999998357          & $2.03\times10^{4}$ & $3.01\times10^{6}$ & $1.64\times10^{-13}$ \\
2  & 2.0000000000550373          & $3.47\times10^{5}$ & $7.38\times10^{19}$ & $5.50\times10^{-11}$ \\
3  & 2.99999999660342            & $2.26\times10^{6}$ & $3.32\times10^{20}$ & $3.40\times10^{-9}$ \\
4  & 4.000000089724362           & $1.05\times10^{7}$ & $8.85\times10^{20}$ & $8.97\times10^{-8}$ \\
5  & 4.99999857388791            & $3.76\times10^{7}$ & $1.84\times10^{21}$ & $1.43\times10^{-6}$ \\
6  & 6.000020476673031           & $1.31\times10^{8}$ & $3.32\times10^{21}$ & $2.05\times10^{-5}$ \\
7  & 6.99960207042242            & $3.94\times10^{8}$ & $5.42\times10^{21}$ & $3.98\times10^{-4}$ \\
8  & 8.007772029099446           & $1.18\times10^{9}$ & $8.29\times10^{21}$ & $7.77\times10^{-3}$ \\
9  & 8.915816367932559           & $2.23\times10^{9}$ & $1.16\times10^{22}$ & $8.42\times10^{-2}$ \\
10 & $10.0954556305 \pm 0.6449i$ & $1.07\times10^{10}$ & $1.72\times10^{22}$ & $6.52\times10^{-1}$ \\
11 & $10.0954556305 \pm 0.6449i$ & $1.07\times10^{10}$ & $1.72\times10^{22}$ & $1.11$ \\
12 & $11.7938905862 \pm 1.6525i$ & $3.14\times10^{10}$ & $2.86\times10^{22}$ & $1.67$ \\
13 & $11.7938905862 \pm 1.6525i$ & $3.14\times10^{10}$ & $2.86\times10^{22}$ & $2.05$ \\
14 & $13.9924066845 \pm 2.5188i$ & $2.16\times10^{11}$ & $4.93\times10^{22}$ & $2.52$ \\
15 & $13.9924066845 \pm 2.5188i$ & $2.16\times10^{11}$ & $4.93\times10^{22}$ & $2.71$ \\
16 & $16.7307448798 \pm 2.8126i$ & $4.85\times10^{11}$ & $8.48\times10^{22}$ & $2.91$ \\
17 & $16.7307448798 \pm 2.8126i$ & $4.85\times10^{11}$ & $8.48\times10^{22}$ & $2.83$ \\
18 & $19.5024423688 \pm 1.9403i$ & $4.56\times10^{12}$ & $1.32\times10^{23}$ & $2.45$ \\
19 & $19.5024423688 \pm 1.9403i$ & $4.56\times10^{12}$ & $1.32\times10^{23}$ & $2.00$ \\
20 & 20.8469102152               & $8.76\times10^{12}$ & $1.59\times10^{23}$ & $0.85$ \\
\hline
\end{tabular}
\caption{Analiza pierwiastków zaburzonego wielomianu Wilkinsona}

\end{table}
W zaburzonym wielomianie widzimy, że niewielka zmiana bardzo znacząco wpłynęła na uzyskane rezultaty. Pojawiła się również część urojona dla pierwiastków od 10 do 19.
\subsection{Wnioski}
W przypadku oryginalnego wielomianu Wilkinsona widzimy, że początkowo pierwiastki są niemal dokładnie wyznaczane, jednak wraz ze wzrostem ich wartości, błedy stają się coraz większe. W rezultacie wartości wielomianu dla tych wyznaczonych miejsc zerowych są bardzo duże i rosną dla coraz większego k, ponieważ te błędy się kumulują. Widzimy zatem jak wyznaczanie pierwiastków Wilkinsona jest źle uwarunkowane. Podobna sytuacja jest przy niewielkim zaburzeniu wartości współczynnika przy $x_{19}$, jednak tutaj błędy są jeszcze większe, a dla pierwiastków od 10 do 19 pojawia się część urojona. Pokazuje to, jak bardzo wielomian ten jest wrażliwy na niewielkie zmiany. Jest to spowodowane ograniczeniami wynikającymi z arytmetyki Float64.
\section{Zadanie 5}
\subsection{Opis problemu}
Zadanie polegało na wykonaniu 40 iteracji zadanego równania rekurencyjnego dla arytmetyki Float64 i Float32 oraz wykonaniu dla Float32 10 iteracji, obcięciu wyniku do 0.722 oraz kontyuacji iteracji aż do 40 wykonań.

Oto równanie rekurencyjne:
\[ p_{n+1} = {p_{n} + rp_{n}(1 - p_{n})} \]
z warunkami:
\[ r = 3 \]
\[ p_0 = 0.01 \]

\subsection{Rozwiązanie}
Kod rozwiązania zadania znajduje się w pliku \texttt{zadanie5.jl}.
\subsection{Wyniki i interpretacja}
Poniższa tabela przedstawia wyniki uzyskane dla poszczególnych przypadków:
\begin{table}[H]
\centering
\begin{tabular}{cccc}
\hline
$n$ & Float32 $p$ & Float64 $p$ & Float32\_cut $p$ \\
\hline
1  & 0.0397 & 0.0397 & -- \\
2  & 0.15407173 & 0.15407173 & -- \\
3  & 0.5450726 & 0.54507263 & -- \\
4  & 1.2889781 & 1.28897800 & -- \\
5  & 0.1715188 & 0.17151914 & -- \\
6  & 0.5978191 & 0.59782012 & -- \\
7  & 1.3191134 & 1.31911379 & -- \\
8  & 0.05627322 & 0.05627158 & -- \\
9  & 0.21559286 & 0.21558684 & -- \\
10 & 0.7229306 & 0.72291430 & -- \\
\hline
11 & 1.3238364 & 1.32384194 & 1.3241479 \\
12 & 0.03771699 & 0.03769530 & 0.03648841 \\
13 & 0.14660022 & 0.14651838 & 0.14195944 \\
14 & 0.52192600 & 0.52167062 & 0.50738037 \\
15 & 1.2704837 & 1.27026177 & 1.2572169 \\
16 & 0.2395482 & 0.24035217 & 0.28708452 \\
17 & 0.7860428 & 0.78810119 & 0.9010855 \\
18 & 1.2905813 & 1.28909430 & 1.1684768 \\
19 & 0.16552472 & 0.17108485 & 0.5778930 \\
20 & 0.5799036 & 0.59652931 & 1.3096911 \\
21 & 1.3107498 & 1.31857559 & 0.09289217 \\
22 & 0.08880425 & 0.05837761 & 0.34568182 \\
23 & 0.3315584 & 0.22328660 & 1.0242395 \\
24 & 0.9964407 & 0.74357568 & 0.94975823 \\
25 & 1.0070806 & 1.31558835 & 1.0929108 \\
26 & 0.9856885 & 0.07003530 & 0.7882812 \\
27 & 1.0280086 & 0.26542635 & 1.2889631 \\
28 & 0.9416294 & 0.85035197 & 0.17157483 \\
29 & 1.1065198 & 1.23211246 & 0.59798557 \\
30 & 0.7529209 & 0.37414649 & 1.3191822 \\
31 & 1.3110139 & 1.07662917 & 0.05600393 \\
32 & 0.0877831 & 0.82912557 & 0.21460639 \\
33 & 0.3280148 & 1.25415465 & 0.72025780 \\
34 & 0.9892781 & 0.29790694 & 1.3247173 \\
35 & 1.0210990 & 0.92538213 & 0.03424144 \\
36 & 0.95646656 & 1.13253226 & 0.13344833 \\
37 & 1.0813814 & 0.68224107 & 0.48036796 \\
38 & 0.81736827 & 1.33260565 & 1.2292118 \\
39 & 1.2652004 & 0.00290916 & 0.3839622 \\
40 & 0.25860548 & 0.01161124 & 1.0935680 \\
\hline
\end{tabular}
\caption{Porównanie wartości $p$ dla Float32, Float64 i Float32\_cut}

\end{table}
Analizując powyższą tabelę, widzimy znaczne różnice w wynikach dla różnych arytmetyk oraz po obcięciu wartości w przypadku Float32.
\subsection{Wnioski}
Dla Float32 i Float64 wyniki są do siebie w miarę podobne aż do 21 iteracji. Póżniej zaczynają znacznie od siebie odbiegać, a Float32 z obcięciem już od 17 iteracji mocno się różni od reszty. Można zatem wysnuć wniosek, że model logistyczny jest bardzo wrażliwy na dokładność obliczeń, a najdrobniejsze zmiany rażąco wpływają na końcowe wyniki. Widać tutaj jak duży wpływ na wynik mają błędy zaokrągleń i precyzja arytmetyki, szczególnie w przypadku iteracyjnych obliczeń, gdzie błąd może się kumulować. Ostateczne wyniki dla 40. iteracji są skrajnie różne dla każdego z przypadków.
\section{Zadanie 6}
\subsection{Opis problemu}
Zadanie polegało na wykonaniu 40 iteracji zadanego równania rekurencyjnego z ustaloną stałą i pierwszą wartością dla Float64 i zaobserwować zachowanie.

Oto równanie rekurencyjne:
\[ x_{n+1} = x_{n}^{2} + c \]
z warunkami:
\begin{enumerate}
    \item $c = -2$ oraz $x_0 = 1$
    \item $c = -2$ oraz $x_0 = 2$
    \item $c = -2$ oraz $x_0 = 1.99999999999999$
    \item $c = -1$ oraz $x_0 = -1$
    \item $c = -1$ oraz $x_0 = 0.75$
    \item $c = -1$ oraz $x_0 = 0.25$
\end{enumerate}

\subsection{Rozwiązanie}
Kod rozwiązania zadania znajduje się w pliku \texttt{zadanie6.jl}. Interpretacja graficzna została wygenerowana z użyciem kalkulatora graficznego GeoGebra.
\subsection{Wyniki i interpretacja}
Poniższe tabele przedstawiają wyniki uzyskane dla poszczególnych iteracji:
\begin{table}[H]
\centering
\begin{tabular}{cccc}
\hline
$n$ & $x_0 = -1.0$ & $x_0 = 0.75$ & $x_0 = 0.25$ \\
\hline
1  & -1.0 & 0.75 & 0.25 \\
2  & 0.0 & -0.4375 & -0.9375 \\
3  & -1.0 & -0.8086 & -0.1211 \\
4  & 0.0 & -0.3462 & -0.9853 \\
5  & -1.0 & -0.8802 & -0.0291 \\
6  & 0.0 & -0.2253 & -0.9992 \\
7  & -1.0 & -0.9492 & -0.0017 \\
8  & 0.0 & -0.0990 & -0.999997 \\
9  & -1.0 & -0.9902 & -5.74e-6 \\
10 & 0.0 & -0.0195 & -1.0 \\
11 & -1.0 & -0.9996 & -6.59e-11 \\
12 & 0.0 & -7.59e-4 & -1.0 \\
13 & -1.0 & -1.0 & 0.0 \\
14 & 0.0 & -1.15e-6 & -1.0 \\
15 & -1.0 & -1.0 & 0.0 \\
16 & 0.0 & -2.66e-12 & -1.0 \\
17 & -1.0 & -1.0 & 0.0 \\
18 & 0.0 & 0.0 & -1.0 \\
19 & -1.0 & -1.0 & 0.0 \\
20 & 0.0 & 0.0 & -1.0 \\
21 & -1.0 & -1.0 & 0.0 \\
22 & 0.0 & 0.0 & -1.0 \\
23 & -1.0 & -1.0 & 0.0 \\
24 & 0.0 & 0.0 & -1.0 \\
25 & -1.0 & -1.0 & 0.0 \\
26 & 0.0 & 0.0 & -1.0 \\
27 & -1.0 & -1.0 & 0.0 \\
28 & 0.0 & 0.0 & -1.0 \\
29 & -1.0 & -1.0 & 0.0 \\
30 & 0.0 & 0.0 & -1.0 \\
31 & -1.0 & -1.0 & 0.0 \\
32 & 0.0 & 0.0 & -1.0 \\
33 & -1.0 & -1.0 & 0.0 \\
34 & 0.0 & 0.0 & -1.0 \\
35 & -1.0 & -1.0 & 0.0 \\
36 & 0.0 & 0.0 & -1.0 \\
37 & -1.0 & -1.0 & 0.0 \\
38 & 0.0 & 0.0 & -1.0 \\
39 & -1.0 & -1.0 & 0.0 \\
40 & 0.0 & 0.0 & -1.0 \\
\hline
\end{tabular}
\caption{Iteracje dla $c = -1$ i różnych wartości początkowych $x_0$}

\end{table}
\begin{table}[H]
\centering
\begin{tabular}{cccc}
\hline
$n$ & $x_0 = 1.0$ & $x_0 = 2.0$ & $x_0 = 1.99999999999999$ \\
\hline
1  & 1.0 & 2.0 & 2.00000000000000 \\
2  & -1.0 & 2.0 & 1.99999999999996 \\
3  & -1.0 & 2.0 & 1.99999999999984 \\
4  & -1.0 & 2.0 & 1.99999999999936 \\
5  & -1.0 & 2.0 & 1.99999999999744 \\
6  & -1.0 & 2.0 & 1.99999999998977 \\
7  & -1.0 & 2.0 & 1.99999999995907 \\
8  & -1.0 & 2.0 & 1.99999999983629 \\
9  & -1.0 & 2.0 & 1.99999999934516 \\
10 & -1.0 & 2.0 & 1.99999999738066 \\
11 & -1.0 & 2.0 & 1.99999998952262 \\
12 & -1.0 & 2.0 & 1.99999995809048 \\
13 & -1.0 & 2.0 & 1.99999983236194 \\
14 & -1.0 & 2.0 & 1.99999932944778 \\
15 & -1.0 & 2.0 & 1.99999731779157 \\
16 & -1.0 & 2.0 & 1.99998927117349 \\
17 & -1.0 & 2.0 & 1.99995708480908 \\
18 & -1.0 & 2.0 & 1.99982834107804 \\
19 & -1.0 & 2.0 & 1.99931339377896 \\
20 & -1.0 & 2.0 & 1.99725404654395 \\
21 & -1.0 & 2.0 & 1.98902372643618 \\
22 & -1.0 & 2.0 & 1.95621538432605 \\
23 & -1.0 & 2.0 & 1.82677862987391 \\
24 & -1.0 & 2.0 & 1.33712016256400 \\
25 & -1.0 & 2.0 & -0.21210967086482 \\
26 & -1.0 & 2.0 & -1.95500948752562 \\
27 & -1.0 & 2.0 & 1.82206209631517 \\
28 & -1.0 & 2.0 & 1.31991028282844 \\
29 & -1.0 & 2.0 & -0.25783684528374 \\
30 & -1.0 & 2.0 & -1.93352016121413 \\
31 & -1.0 & 2.0 & 1.73850021382151 \\
32 & -1.0 & 2.0 & 1.02238299345744 \\
33 & -1.0 & 2.0 & -0.95473301468901 \\
34 & -1.0 & 2.0 & -1.08848487066284 \\
35 & -1.0 & 2.0 & -0.81520068633810 \\
36 & -1.0 & 2.0 & -1.33544784099389 \\
37 & -1.0 & 2.0 & -0.21657906398475 \\
38 & -1.0 & 2.0 & -1.95309350904349 \\
39 & -1.0 & 2.0 & 1.81457425506782 \\
40 & -1.0 & 2.0 & 1.29267972715492 \\
\hline
\end{tabular}
\caption{Iteracje dla $c = -2$ i różnych wartości początkowych $x_0$}

\end{table}
\begin{figure}[h!]
    \centering
    \includegraphics[width=0.7\textwidth]{screen1.jpg}
    \caption{Wykres dla podpunktu 1}
    \label{fig:wykres}
\end{figure}
\begin{figure}[h!]
    \centering
    \includegraphics[width=0.7\textwidth]{screen2.jpg}
        \caption{Wykres dla podpunktu 2}

    \label{fig:wykres}
\end{figure}\begin{figure}[h!]
    \centering
    \includegraphics[width=0.7\textwidth]{screen3.jpg}
        \caption{Wykres dla podpunktu 3}

    \label{fig:wykres}
\end{figure}\begin{figure}[h!]
    \centering
    \includegraphics[width=0.7\textwidth]{screen4.jpg}
        \caption{Wykres dla podpunktu 4}

    \label{fig:wykres}
\end{figure}\begin{figure}[h!]
    \centering
    \includegraphics[width=0.7\textwidth]{screen5.jpg}
        \caption{Wykres dla podpunktu 5}

    \label{fig:wykres}
\end{figure}\begin{figure}[h!]
    \centering
    \includegraphics[width=0.7\textwidth]{screen6.jpg}
        \caption{Wykres dla podpunktu 6}

    \label{fig:wykres}
\end{figure}\begin{figure}[h!]
    \centering
    \includegraphics[width=0.7\textwidth]{screen7.jpg}
        \caption{Wykres dla podpunktu 7}

    \label{fig:wykres}
\end{figure}
\clearpage
\subsection{Wnioski}
Dla podpunktu 1 i 2 widzimy, że kolejne iteracje nie wpływają na zmianę wartości - pozostają one stałe. Graficzne iteracje dla tych przypadków prowadzą do zachowania stabilnego. W podpunkcie 3. mimo dużego podobieństwa do podpunktu 2, widzimy, że uzyskane wartości są znacznie różne. Z kolejnymi iteracjami wyniki coraz bardziej odbiegają od początkowej wartości. Następne podpunkty, gdzie stała c = -1, z każdą kolejną iteracją wyniki zaczynają się coraz bardziej stabilizować, wchodząc w cykl (0, -1). W tym wypadku wartość x0 decyduje jedynie o tym, jak szybko osiągnięty zostanie tenże cykl.
\end{document}